\documentclass[12pt,a4paper]{report}
\usepackage{graphicx}
\usepackage{caption}
\usepackage{geometry}
\usepackage{listings}
 \geometry{
 a4paper,
 top=20mm,
 }
\usepackage[pdftex]{hyperref}
\usepackage{float}
\usepackage{titlesec}
\usepackage[utf8]{inputenc}
\usepackage[portuges]{babel}
\titleformat{\chapter}{\normalfont\huge}{\thechapter.}{10pt}{\huge\it} 
\begin{document}
\begin{titlepage}
	\centering
	\includegraphics[width=0.3\textwidth]{uminho.jpg}\par\vspace{1cm}
	{\huge\bfseries Sistemas Operativos \par}
	\vspace{0.5cm}
	{\scshape\ MIEI - 2º ano - 2º semestre\par}
	\vspace{0.1cm}
	{\scshape\ Universidade do Minho\par}
	\vspace{1.5cm}
    {\scshape\Huge\bfseries Sistema de Backup \par}
	\vspace{5cm}
    \vspace{1cm}
	{\scshape\ Bruno Cancelinha \par} 	\vspace{0.1cm}
	{\scshape\ A75428 \par}  \vspace{0.3cm}
	{\scshape\ Marcelo Miranda \par} \vspace{0.1cm}
	{\scshape\ A74817 \par}  \vspace{0.3cm}
	{\scshape\ Rui Vieira \par} \vspace{0.1cm}
	{\scshape\ A74658 \par}  \vspace{0.3cm}

	\vfill
	{\large \today\par}
\end{titlepage}

\tableofcontents

\chapter{Introdução}

Este projeto foi realizado no âmbito da disciplina de \emph{Sistemas Operativos} e tem como objetivo a criação de um sistema de cópias eficiente, que guarda ficheiros dados por um utilizador. Estes são então comprimidos, reduzindo o espaço por eles ocupados. Temos também de considerar a privacidade de dados mantendo uma arquitetura cliente/servidor impedindo o acesso direto do cliente à pasta de backup.\par
O trabalho parecia ser fácil à primeira vista mas o nosso grupo não estava a considerar toda a dificuldade duma arquitetura com processos concorrentes, o que nos levou a adquirir uma nova maneira de pensar.\par
O projeto tem então duas funcionalidades principais. O \emph{\bfseries{backup}} que se responsabiliza por comprimir os ficheiros e salva-los na pasta que viremos a chamar \emph{raiz do backup}. O \emph{\bfseries{restore}} que simplesmente descomprime o ficheiro e o devolve na sua diretoria original. Vamos de seguida explicar estes dois com mais profundidade.

\chapter{Funcionalidades}
\section{Backup}
A funcionalidade \emph{backup} é a principal de todo o trabalho. Aos olhos do utilizador apenas guarda o ficheiro ou todo o conteúdo de uma pasta, mas visto de mais perto, é bem mais complexo.\par Primeiro, o cliente terá que enviar todo o conteúdo do ficheiro a salvar em blocos de 4kbytes até o ficheiro estar completamente transferido para o servidor. Este terá então de lhe atribuir um \emph{digest} gerado pelo \emph{sha1sum}, comprimi-lo na pasta \emph{data}  usando o comando \emph{gzip} e alterar o seu nome para esse \emph{digest}. É também guardado na pasta \emph{metadata} um \emph{link simbólico} com o nome original do ficheiro ligado ao ficheiro correspondente em \emph{data}, para além de criar outro \emph{link simbólico} na pasta \emph{paths} ligado à diretoria original desse ficheiro com o \emph{path} original do ficheiro, para que este possa ser corretamente recuperado mais tarde. Quando o backup estiver concluído o servidor envia um sinal de sucesso ou de erro ao cliente.

\section{Restore}
Esta funcionalidade complementa o backup, permitindo-nos reaver os ficheiros guardados.\par O restore começa por ler dos \emph{links simbólicos} do \emph{metadata} o \emph{digest} correspondente ao conteúdo que pretendemos recuperar da pasta \emph{data}. A partir deste, é criada uma cópia do conteúdo para que possamos descomprimir o ficheiro sem comprometer futuros \emph{restores}. Após ser descomprimido, o conteúdo é enviado para o cliente, em conjunto com o \emph{path} original do ficheiro, este é por fim, lá montado.

\section{Delete}
O comando \emph{delete} apaga a entrada do ficheiro da \emph{raiz do backup}. Para isso, apenas apaga o ficheiro de \emph{metadata/} e de \emph{paths}, mantendo o conteúdo comprimido em \emph{data/}.

\section{Global Clean}
O \emph{Global Clean} é chamado pelo nome de \emph{gc}, remove todos os conteúdos em \emph{data/} que não estão a ser ligados por nenhum ficheiro em \emph{metadata/}.

\chapter{Makefile}
\section{make}
Compila o cliente para o ficheiro \emph{client} e servidor para \emph{server}.

\section{make clean}
Apenas limpa os executáveis criados pelo \emph{make}.

\section{make stop}
Para todos os processos de \emph{sobusrv}.

\section{make install}
Instala o \emph{sobucli} e \emph{sobusrv}, deve ser executado depois de fazer \emph{make}. Este comando irá necessitar de permições \emph{sudo} pois instala estes executáveis diretamente na pasta \emph{/bin}. O nosso grupo questionou-se sobre colocar na diretoria \emph{/bin} ou alterar o \emph{PATH} de \emph{.bashrc}, acabamos por escolher a primeira pois a segunda alternativa não iria funcionar com nenhum de nós visto que o nosso path está guardado em \emph{.zshrc} do \emph{zsh}, uma shell alternativa à \emph{bash} habitual. Mantemos então o instalador a copiar para a pasta \emph{/bin} para manter compatibilidade.

\section{make uninstall}
Pareceu-nos importante, depois de ter um instalador, ter também um desinstalador. O make uninstall apenas remove \emph{sobusrv} e \emph{sobucli} da pasta \emph{/bin}.

\chapter{Raiz do Backup}
A \emph{raiz do backup} é uma pasta que se encontra na \emph{home} do utilizador que corre o servidor. Dentro dela, encontra-se o \emph{pipe} que servirá de comunicação cliente/servidor e, ocasionalmente, um conjunto de outros \emph{pipes} para servir de comunicação servidor/cliente necessária para o comando \emph{restore}, também uma pasta \emph{data/}, \emph{metadata/} e \emph{paths}.\par

\section{data/}
Na pasta \emph{data/} encontra-se todos os ficheiros comprimidos com o comando \emph{gzip}, para além disso, o nome dos ficheiros são o seu \emph{digest} criado por \emph{sha1sum}. Deste modeo é fácil descobrir ficheiros repetidos quando é chamado um novo \emph{backup}. O comando \emph{delete} não apaga nenhum destes ficheiros, para isso encontra-se designado o \emph{gc} que limpa todos os ficheiros de \emph{data/} que não estão ligados por \emph{metadata/}.

\section{metadata/}
Na pasta \emph{metadata/} estão guardados os \emph{links simbólicos} que ligam o nome do ficheiro ao seu conteúdo em \emph{data/}. O comando \emph{delete} apenas apaga estes \emph{links}.

\section{paths/}
Tal como a pasta \emph{metadata/}, a pasta \emph{data/} contém \emph{links simbólicos} que ligam o nome do ficheiro ao seu \emph{path} original para depois ser usado no \emph{restore}.

\chapter{Comunicação cliente/servidor}

\section{Pipe do servidor}
O servidor cria um pipe que se encontra na \emph{raiz do backup} com o nome \emph{sobupipe}. O servidor fica numa espera passiva até que o cliente transmita as operações que pretende que o primeiro execute. Estes pedidos são enviados em forma de \emph{mensagem}, uma estrutura que iremos discutir de seguida.

\section{Sinais}
Usamos sinais principalmente para o servidor notificar o cliente que o seu ficheiro já foi processado. Para sucesso, o servidor envia \emph{SIGUSR1} para erro envia \emph{SIGUSR2}. O utilizador ao receber cada um dos sinais escreve no ecrã a mensagem correspondente. Conforme o exemplo:
\begin{lstlisting}
a.txt: copiado
b.txt: erro ao copiar
\end{lstlisting}

\section{Pipe do cliente}
Ao fazer o \emph{restore} é necessário um outro \emph{pipe} que transfira, com estruturas do tipo mensagem, o documento descomprimido para o cliente, que o monta na sua localização original. Será criado na \emph{raiz do backup} pelo servidor com o nome que do \emph{pid} do processo que lhe enviou o pedido. Este \emph{pipe} é imediatamente removido depois de já não ser necessário.  

\chapter{Mensagem}

\begin{lstlisting}
#define CHUNK_SIZE 4096
#define PATH_SIZE 1024

#define BACKUP 0
#define RESTORE 1
#define DELETE 2
#define CLEAN 3

#define NOT_FNSHD 1
#define FINISHED  0 
#define ERROR -1

typedef struct message {
    char chunk[CHUNK_SIZE];
    char file_path[PATH_SIZE];
    int operation;
    int status;
    int chunk_size;
    pid_t pid;
    uid_t uid;
} *MESSAGE;
\end{lstlisting}

\section{A estrutura}

\hspace{0cm}\par
Um \emph{\bfseries{chunk}} é um \emph{array} de 4kbytes que terá \emph{\bfseries{chunk\_size}} bytes do ficheiro a transferir pelos \emph{pipes}. \par
O \emph{path} do ficheiro está guardado na \emph{String} \emph{\bfseries{file\_path}}. \par
A operação a efetuar está especificada no inteiro \emph{\bfseries{operation}}. Existem defines para cada tipo de operação, portanto o \emph{\bfseries{operation}} pode estar para \emph{BACKUP}, \emph{RESTORE}, \emph{DELETE} ou \emph{CLEAN}. \par
O inteiro \emph{\bfseries{status}} comunica o estado do ficheiro, este pode ser \emph{NOT\_FINISHED} caso faltem mais \emph{chunks} para carregar o ficheiro, \emph{FINISHED} caso tenha terminado de carregar todo o ficheiro, e \emph{ERROR} caso tenha ocorrido um erro na leitura do ficheiro.\par
Finalmente temos o \emph{\bfseries{pid}} do processo que mandou a mensagem.

\section{empty\_message}
Apenas aloca o espaço para uma nova mensagem. Os campos desta nova mensagem não esterão tratados, sendo, por tanto, valores aleatórios.\par

\section{init\_message}
Para além de alocar uma nova mensagem, preenche todos os seus campos com os valores passados nos argumentos.\par

\section{change\_message}
Altera uma mensagem dada. Para tal, a mensagem passada nos argmuentos deve estar inicializada, sendo assim populada com os valores passados pelos argumentos.

\section{freeMessage}
Liberta o espaço alocado em memória pela mensagem.


\chapter{Conclusão}
Este projeto desenvolveu-nos uma capacidade de pensar numa arquitetura com vários processos concorrentes  para além de nos deixar mais à vontade com o sistema Unix e a sua interação com a linguagem C.\par
Apesar de não estar em desagrado com o nosso trabalho, há algumas funcionalidas que gostariamos de ter implementado.\par
A habilidade de poder gravar ficheiros com o mesmo nome, por exemplo, o ficheiro \emph{a.txt} da pasta \emph{~/Desktop} e outro como o mesmo nome em \emph{~/Documents}. Que não nos foi possível efeturar devido à ambiguidade que provinha de chamar \emph{restore} no tal ficheiro \emph{a.txt}, mantendo a simplicaidade do comando.\par
Seria também interessante a possibilidade de manter as várias versões do mesmo ficheiro, apresentando um histórico ao utilizador cada vez que este fosse correr o \emph{restore} nesse ficheiro, tornando este mais poderoso. Não se concretizou para manter a simplicidade do programa.\par
Finalmente, o cliente deveria calcular o sha1sum, assim não seria necessário transferir o ficheiro todo para o servidor para de seguida concluir que já existe uma versão deste lá comprimida. A princípio pensamos que talvez não fosse correto o cliente executar comandos (neste caso o sha1sum), quando decidimos que o devia fazer e enviar para o servidor o \emph{digest}, já foi muito em cima da hora e decidimo-nos focar em aspetos mais importantes.\par
Foi sem dúvida um trabalho bastante interessante que nos despertou um grande interesse no ramo de \emph{Sistemas Operativos}.


\end{document}